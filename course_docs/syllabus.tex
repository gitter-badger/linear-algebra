\documentclass[10pt]{article}
\usepackage{url}
\usepackage{paralist}
\begin{document}

\begin{center}
{\Large Math 2500: Linear Algebra\\
Spring 2014, Section 01}
\end{center}

\section*{Basic Information:}
\begin{tabular}{ll}
Instructor: & Theron J Hitchman \\
Office: & 327 Wright Hall\\
email:	 & theron.hitchman@uni.edu \\
phone: & x3-2646 \\
office hours:	& Drop-in MWF 11am to noon \\
		& or make an appointment (I am happy to do this.) \\
\end{tabular}

\section*{Course Description:}
	At the beginning, linear algebra is the study of vectors and matrices. The subject grows from there and is the first place that most students encounter seriously modern flavors of abstraction in mathematics. This is what gives the subject its power, and what makes it challenging for the newcomer. The official catalog description for the course is as follows:
\begin{quotation}
``Gaussian elimination; matrix algebra; vector spaces, kernels, and other subspaces; orthogonal projection; eigenvalues and eigenvectors.''
\end{quotation}
That is all fine, and we will do that stuff. But to make it work, we will focus on two other things, too:
\begin{compactitem}
\item careful use of mathematical terms
\item making and exploring examples
\end{compactitem}

\section*{Expectations \& Course Structure:}
	Class meetings will be very active. I will lecture rarely! For each class meeting there will be a short reading assignment with some associated routine homework exercises. (I will be using a structure called ‘’guided practice’’ where assignments will also contain explicit learning goals to tackle during your study.) Class meetings will begin with a 5 minute quiz to check comprehension. (These will be given feedback, but will not be graded.) During class meetings we will run a variety of activities and discussions to explore the material further. You need to come prepared for meetings, or you will not get much out of them.

\subsection*{The Mailing List:}
	In lieu of a course web page, we shall use the university created Google Groups Mailing List to communicate. I will make announcements to this list. Students should ask questions related to class on this list. I have set it so all are free to post. The url is:
\url{https://groups.google.com/a/uni.edu/forum/#!forum/math-2500-01-spring}

\subsection*{Course Materials:}
\begin{tabular}{ll}
Required Text: & \emph{Introduction to Linear Algebra}, 4th Ed by Gilbert Strang \\
                        & available at UBS, or through many online retailers \\

Optional Text: & \emph{Linear Algebra}, by Jim Hefferon\\
                       & available online at \url{http://joshua.smcvt.edu/linearalgebra/} \\
\end{tabular}

There are many good linear algebra texts, and I am sure you could learn a lot from them, too. I will make assignments out of Strang’s book, but you can easily profit by looking at the discussions of the same topics in another introductory book.

\subsection*{Computational Software:}
	Computing power is important for modern mathematics, especially in a subject as computationally intensive as linear algebra. Even the simplest computations for very large matrices are unworkable with a pencil and paper. These days there are many options for getting a computer to do the work for you. My favorite is Sage. The main benefits of Sage are that it is open-source, and free. We will discuss how to use Sage in class, we will use the cloud service at \url{https://cloud.sagemath.com/}.

\subsection*{Evaluation:}
	Student grades will be based on four exams: three during the course of the term, and the final. Each exam will be worth 25\% of the course grade. I reserve the right to reweigh these in a student’s favor to account for growth.

\subsection*{Pacing:}
	I expect we shall discuss the first seven chapters of Strang’s book, and some of chapter 8. This is a pretty good clip, so I reserve the right to streamline as needed.

\subsection*{}
\emph{Please address any special needs or special accommodations with me at the beginning of the semester or as soon as you become aware of your needs. Those seeking accommodations based on disabilities should obtain a Student Academic Accommodation Request (SAAR) form from Student Disability Services (SDS) (phone 319-273-2677, for deaf or hard of hearing, use Relay 711). SDS is located on the top floor of the Student Health Center, Room 103.}

\end{document}

