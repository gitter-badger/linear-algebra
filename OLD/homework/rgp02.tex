\documentclass[12pt,letterpaper]{article}

\usepackage[utf8]{inputenc}
\usepackage[T1]{fontenc}
\usepackage{amsmath}
\usepackage{amsfonts}
\usepackage{amssymb}
\usepackage{amsthm}
\usepackage[left=2cm,right=2cm,top=2cm,bottom=2cm,headheight=22pt]{geometry}
\usepackage{fancyhdr}
\usepackage{setspace}
\usepackage{lastpage}
\usepackage{graphicx}
\usepackage{caption}
\usepackage{subcaption}
\usepackage{paralist}
\usepackage{url}

\theoremstyle{definition}
\newtheorem{question}{Question}
\newtheorem{example}{Example}
\newtheorem{exercise}[question]{Exercise}
\newtheorem*{challenge}{Challenge}
\newtheorem*{theorem}{Theorem}
\newtheorem*{definition}{Definition}

\begin{document}

%Paramètres de mise en forme des paragraphes selon les normes françaises
\setlength{\parskip}{1ex plus 0.5ex minus 0.2ex}
\setlength{\parindent}{0pt}

%Paramètres relatifs aux en-têtes et pieds de page.
\pagestyle{fancy}
\lhead{Theron J Hitchman}
\chead{\Large Reading and Guided Practice \# 2}
\rhead{Spring 2014}
\lfoot{\emph{Linear Algebra}}
\cfoot{}
\rfoot{\emph{\thepage\ of \pageref{LastPage}}}

\section*{Introduction}
This assignment is designed to prepare you for our third class meeting.  Complete it before our meeting on Wednesday, January 22.

\section*{Goals}
Before the next class meeting, a student should be able to:
\begin{compactitem}
\item Compute the dot product of two vectors.
\item Compute the norm (length) of a vector.
\item explain the connection between the norm of a vector and the Pythagorean theorem.
\item Compute the angle between two vectors.
\item ``Normalize'' a given vector. That is, find a unit vector pointing in the same direction as a given vector.
\end{compactitem}
After the next class meeting, a student should be able to:
\begin{compactitem}
\item Explain the connections between dot products and the equations of hyperplanes.
\begin{compactitem}
\item Find a vector normal to a given hyperplane.
\item Find the equation of a hyperplane normal to a given vector and through a given point.
\end{compactitem}
\item Describe level sets of the dot product with a given vector.
\end{compactitem}

\section*{Reading}
\begin{compactdesc}
\item[Required:] \emph{Strang} Section 1.2, pages 11-18.
\item[Optional:] \emph{Hefferon} Chapter One: Systesm of Equations, Part II Linear Geometry, section II.2 Length and Angle Measures, pages 39-43.
\end{compactdesc}

\section*{Exercises}
The minimal exercise set is Strang, Section 1.2 exercises 1, 6, 7, 8, 12, 13.

\emph{Note: Strang's text has wonderful exercises. A student with the time to do it would do well to complete more exercises than listed here.}


\section*{Sage Playtime}

In the set of Sage files I have given you, explore these:
\begin{compactitem}
\item SageBeginnerTutorial.sagews
\item meeting02-lines-and-planes-in-sage.sagews
\end{compactitem}
Both can be found in the directory \texttt{Hitchman/Sage\_help}.

%\begin{thebibliography}{9}
%\end{thebibliography}

\end{document}
%sagemathcloud={"zoom_width":100}