\documentclass[12pt,letterpaper]{article}

\usepackage[utf8]{inputenc}
\usepackage[T1]{fontenc}
\usepackage{amsmath}
\usepackage{amsfonts}
\usepackage{amssymb}
\usepackage{amsthm}
\usepackage[left=2cm,right=2cm,top=2cm,bottom=2cm,headheight=22pt]{geometry}
\usepackage{fancyhdr}
\usepackage{setspace}
\usepackage{lastpage}
\usepackage{graphicx}
\usepackage{caption}
\usepackage{subcaption}
\usepackage{paralist}
\usepackage{url}

\theoremstyle{definition}
\newtheorem{question}{Question}
\newtheorem{example}{Example}
\newtheorem{exercise}[question]{Exercise}
\newtheorem*{challenge}{Challenge}
\newtheorem*{theorem}{Theorem}
\newtheorem*{definition}{Definition}

\begin{document}

%Paramètres de mise en forme des paragraphes selon les normes françaises
\setlength{\parskip}{1ex plus 0.5ex minus 0.2ex}
\setlength{\parindent}{0pt}

%Paramètres relatifs aux en-têtes et pieds de page.
\pagestyle{fancy}
\lhead{Theron J Hitchman}
\chead{\Large Reading and Guided Practice \#1}
\rhead{Spring 2014}
\lfoot{\emph{Linear Algebra}}
\cfoot{}
\rfoot{\emph{\thepage\ of \pageref{LastPage}}}

\section*{Introduction}
This assignment is designed to prepare you for our second class meeting.  Complete it before our meeting on Wednesday, January 15th.

\section*{Goals}
Before the next class meeting, a student should be able to:
\begin{compactitem}
\item Represent a vector as a list of numbers, a point in space, or an arrow.
\item Use the vector operations of \emph{addition}, \emph{scalar multiplication}, and \emph{linear combination}.
\item Draw representations of the vector operations, for sure with paper and pencil, perhaps with a computer.
\end{compactitem}
After the next class meeting, a student should be able to:
\begin{compactitem}
\item Connect vector equations $a\cdot\mathbf{u} + b\cdot \mathbf{v} = \mathbf{w}$ involving linear combinations of vectors to two types of pictures: one using the components to form the equations of \emph{hyperplanes} (the row picture), and one using the linear combination directly (the column picture).
\end{compactitem}

\section*{Reading}
\begin{compactdesc}
\item[Required:] \emph{Strang} Chapter 1 section 1 (pages 1--7).
\item[Optional:] \emph{Hefferon} Chapter One, Part II Linear Geometry, Section 1 Vectors in Space (pages 32--37).
\end{compactdesc}

\section*{Exercises}
The minimal exercise set is Strang, Section 1.1 exercises 1,3,5,6,16,20.

\emph{Note: Strang's text has wonderful exercises. A student with the time to do it would do well to complete more exercises than listed here.}


%\begin{thebibliography}{9}
%\end{thebibliography}

\end{document}
%sagemathcloud={"zoom_width":100}