\documentclass[11pt]{amsart}

\usepackage[margin=1in]{geometry}

\theoremstyle{definition}
\newtheorem{task}{Task}

\begin{document}

\title{Meeting 02, Group Work}
\author{Linear Algebra section 01, Spring 2014}

\maketitle

\emph{Basic Advice: Successful students in linear algebra learn to speak with abstract terminology, compute with algebra, and think with geometry. If you feel stuck, try drawing a picture.}

\vspace{1cm}

The focus of today's seminar work is to start working and thinking in three dimensional space. To that end, I have the following tasks for you to work on. Let's see how far we can go today.

\vspace{.5cm}

\begin{center}\textbf{Part One: 2D Warm Up}\end{center}

\begin{task}
Consider the line in the Cartesian plane described in parametric vector form as
\[
\ell = \left\{ X = \left(\begin{smallmatrix} 2 \\ 1 \end{smallmatrix}\right)
+ t \left( \begin{smallmatrix} -1 \\ 1 \end{smallmatrix} \right) \mid \text{$t$ is a real number}\right\}.
\]

Write the pair of parametric equations for the coordinates $x$ and $y$ of a generic point $X = \left( \begin{smallmatrix} x \\ y \end{smallmatrix}\right)$ on $\ell$ as functions of the parameter $t$.
\end{task}

\vspace{1.5in}

\begin{task}
Eliminate the parameter $t$ from the equations to write the standard form of the equation of $\ell$.
\end{task}

\vfill

\clearpage

\begin{center}\textbf{Part Two: working in 3D}\end{center}

For the rest of this investigation, we shall consider the two points $P$ and $Q$ in $\mathbb{R}^3$ given by the vectors
\[
P = \begin{pmatrix} 3 \\ -2 \\ 0\end{pmatrix} \qquad \text{and} \qquad
Q = \begin{pmatrix} 1 \\ 1 \\ 1 \end{pmatrix}.
\]

\begin{task}
Write the vector parametric form of the line $m$ in $\mathbb{R}^3$ which passes through $P$ and $Q$.
\end{task}

\vspace{1.5in}

\begin{task}
Find some equations in the coordinates $x, y$ and $z$ of a generic point $X = \left( \begin{smallmatrix} x \\ y \\ z\end{smallmatrix} \right)$ which describe exactly when the point $X$ lies on the line $m$.
\end{task}

\vspace{1.5in}

\begin{task} Draw a picture that represents the geometry of what you just found.
\end{task}

\clearpage

\begin{center}\textbf{Part III: A linear combination equation}\end{center}

Now consider these four vectors in $\mathbb{R}^3$.
\[
u = \begin{pmatrix} 2 \\ -1 \\ 0 \end{pmatrix}, \quad v = \begin{pmatrix} -1\\2\\-1 \end{pmatrix}, \quad w = \begin{pmatrix}0\\-1\\2\end{pmatrix}, \quad b = \begin{pmatrix} 1\\0\\0\end{pmatrix}
\]


\begin{task}[\emph{Strogatz}, p 10 \#31]
Write down three equations for $c$, $d$, and $e$ so that $cu + dv + ew = b$.
\end{task}

\vspace{2in}

\begin{task}
Can you somehow find $c$, $d$, and $e$?
\end{task}


\end{document}
%sagemathcloud={"zoom_width":100}























