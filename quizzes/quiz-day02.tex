\documentclass[11pt]{amsart}
\usepackage[margin=1in]{geometry}
\usepackage{paralist}
\usepackage{multicol}

\theoremstyle{definition}
\newtheorem{problem}{Problem}

\begin{document}

\title{Quiz, Meeting 02}
\author{Linear Algebra Section 01, Spring 2014}

\maketitle

\begin{problem}
Suppose that we have the two vectors $u$ and $v$ given below. What is the vector $2u - 3v$?
\[
u = \begin{pmatrix} 2 \\ 1 \\ -10 \end{pmatrix}, \qquad 
v = \begin{pmatrix} -5 \\ 3 \\ 7\end{pmatrix}. \qquad \text{soln: } 2 u-3v = 
\]

\end{problem}

\vspace{.25in}

\begin{problem}
I've left some space here. Treat this space as a piece of the Cartesian coordinate plane, but \emph{don't draw coordinate axes on it}! Your piece of paper is very far out in the first quadrant, and you can't see those axes. They lie off the page.\\

In the space, pick and label two points, called $M$ and $N$. (Don't make them too close together or too far apart.) Add to your diagram the following points as accurately as you can. Be sure to label them clearly.
\begin{compactitem}
\begin{multicols}{3}
\item $(M+N)/2$,
\item $\frac{1}{4}M + \frac{3}{4}N$,
\item $2 M - N$.
\end{multicols}
\end{compactitem}

\end{problem}

\vfill




\end{document}