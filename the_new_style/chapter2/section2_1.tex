\documentclass[11pt]{amsart}
\usepackage[margin=1in]{geometry}
\usepackage{paralist}
\usepackage{graphicx}
\usepackage{mathtools}

\theoremstyle{definition}
\newtheorem{exercise}{Exercise}

\begin{document}
\title{Linear Algebra}
\author{Strang, Section 2.1}
\maketitle

\section{The assignment}
\begin{compactitem}
\item Read section 2.1 of Strang (pages 31-40).
\item Read the following and complete the exercises below.
\end{compactitem}


\section{Some Notes on the Three Perspectives}

Now we have built a little experience with vectors and related things, it is time to be aware of what we have done so we can use it as a foundation for future work.

Does the following picture make sense to you?


\framebox{
\begin{tabular}{l}
\textbf{Row Picture}\\
Algebra: system of linear equations\\
Geometry: (hyper)planes intersecting
\end{tabular}
} $\longleftrightarrow$
\framebox{
\begin{tabular}{l}
\textbf{Column Picture}\\
Algebra: equation on column vectors\\
Geometry: linear combination of vectors
\end{tabular}
}
\begin{center}
$\searrow\nwarrow \hspace{1in} \swarrow\nearrow$\\
\framebox{
\begin{tabular}{l}
\textbf{Transformational Picture}\\
Algebra: matrix equation\\
Geometry: ????
\end{tabular}
}
\end{center}

A deep understanding of linear algebra will involve a level of comfort with each of the three views of the subject in the diagram, and also the ability to pass back and forth between them.

\subsection{The Transformational View}

We have seen that matrices can be made to ``act upon'' vectors by a kind of multiplication. In particular, if $A$ is an $m \times n$ matrix, then $A$ can be multiplied (on the left) with a column vector of size $m$, and the result is a column vector of size $n$.

This makes $A$ into a kind of \emph{function}. (We will use the synonyms \emph{mapping} or \emph{transformation}, too.)
For every vector $v$ of size $m$, the matrix $A$ allows us to compute a new vector $T_A(v) = Av$ of size $n$. This is the basic example of what we will eventually call a \emph{linear transformation}.
\begin{align*}
\mathbb{R}^m & \xrightarrow[]{T_A} \mathbb{R}^n \\
v & \xmapsto[]{\phantom{T_A}} Av
\end{align*}
One of our long term goals is to find a way to think about the geometry of linear algebra from this viewpoint.



\section{Sage instructions}

I have made a Sage worksheet file with some basic commands that you might find useful in investigating linear systems. The file is called \texttt{section2\_1.sagews}.


\section{Questions for Section 1.1}

\begin{exercise}
Find an example of a vector $b$ so that the equation
\[
\begin{pmatrix} -1 & 2 \\ 5 & -9\end{pmatrix} v = b
\]
has no solution $v$, or explain why it is impossible to find such an example.
\end{exercise}

\begin{exercise}
Consider the matrix equation
\[
\begin{pmatrix} 1 & 2 & 4 \\ 2 & 0 & 1 \end{pmatrix} \begin{pmatrix} x \\ y \\ z \end{pmatrix} = \begin{pmatrix} -1 \\ 3 \end{pmatrix} .
\]
\begin{compactitem}
\item[a)] Draw a diagram representing the row picture of this equation.
\item[b)] Draw a diagram representing the column picture of this equation.
\item[c)] Draw a diagram representing the transformational picture of this equation.
\end{compactitem}
\end{exercise}

\begin{exercise}
Find an example of a matrix $B$ which has the following effect:
\begin{compactitem}
\item[a)] $B \left(\begin{smallmatrix} x \\ y \end{smallmatrix}\right) = \left(\begin{smallmatrix} y \\ x \end{smallmatrix}\right)$
\item[b)] Rotates vectors through $45^{\circ}$ counter-clockwise.
\item[c)] Reflects vectors across the $y$-axis.
\item[d)] $B \left(\begin{smallmatrix} x \\ y \end{smallmatrix}\right) = \left(\begin{smallmatrix} X+y \\ y \end{smallmatrix}\right)$
\end{compactitem}
\end{exercise}


\end{document}




%sagemathcloud={"zoom_width":100}