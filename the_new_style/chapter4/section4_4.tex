\documentclass[11pt]{amsart}
\usepackage[margin=1in]{geometry}
\usepackage{paralist}
\usepackage{graphicx}
\usepackage{mathtools}

\theoremstyle{definition}
\newtheorem{exercise}{Exercise}
\newtheorem*{theorem}{Theorem}
\newtheorem*{definition}{Definition}

\begin{document}
\title{Linear Algebra}
\author{Strang, Section 4.4}
\maketitle

\section{The assignment}
\begin{compactitem}
\item Read section 4.4 of Strang (pages 230-238).
\item Read the following.
\item Prepare the items below for presentation.
\end{compactitem}


\section{Orthonormal bases, orthogonal matrices, and the Gram-Schmidt Algorithm}

There are four main points to take away from this section:

\begin{itemize}
\item The idea of an orthonormal basis.

\item The idea of an orthogonal matrix. The special property that $Q^T = Q^{-1}$ for an orthogonal matrix $Q$.

\item The Gram-Schimdt algorithm for constructing an orthonormal basis.

\item The $QR$ decomposition of a matrix $A$.
\end{itemize}


\section{Sage instructions}

I have made a Sage worksheet file with some basic commands that you might find useful. The file is called \texttt{section4\_4.sagews}.


\section{Questions for Section 4.4}
\setcounter{exercise}{130}

Please do exercises 12 and 18 to be sure you feel comfortable with the basic idea behind the Gram-Schmidt algorithm. I don't plan to discuss these unless everyone feels lost on them. In particular, exercise 12 shows one reason why one might prefer an orthonormal basis to just any old basis.


\begin{exercise}
Exercise 18 from section 4.4 of Strang.
\end{exercise}

\begin{exercise}
Exercise 19 from section 4.4 of Strang.
\end{exercise}

\begin{exercise}
Exercise 21 from section 4.4 of Strang.
\end{exercise}

\begin{exercise}
Exercise 23 from section 4.4 of Strang.
\end{exercise}

\begin{exercise}
Exercise 24 from section 4.4 of Strang.
\end{exercise}

\end{document}




%sagemathcloud={"zoom_width":100}