\documentclass[11pt]{amsart}
\usepackage[margin=1in]{geometry}
\usepackage{paralist}
\usepackage{graphicx}
\usepackage{mathtools}

\theoremstyle{definition}
\newtheorem{exercise}{Exercise}
\newtheorem*{theorem}{Theorem}
\newtheorem*{definition}{Definition}

\begin{document}
\title{Linear Algebra}
\author{Strang, Section 4.1}
\maketitle

\section{The assignment}
\begin{compactitem}
\item Read section 4.1 of Strang (pages 195-202).
\item Read the following and complete the exercises below.
\end{compactitem}


\section{Orthogonality of The Four Subspaces}


Previously, we had the notion of orthogonality for two vectors in Euclidean space. In this section, the concept gets extended to subspaces.
\begin{definition}
Let $V$ and $W$ be subspaces of $\mathbb{R}^n$. We say that $\mathbb{R}^n$ and $\mathbb{R}^m$ are \emph{orthogonal} when for each vector $v \in V$ and each vector $w \in W$ we have $v \cdot w =0$.
\end{definition}

Two orthogonal subspaces always have as intersection the trivial subspace $\{ 0\}$. The reason for this is that if some vector $x$ lay in both $V$ and $W$, then we must have that $x \cdot x =0$. (Think of the first $x$ as lying in $V$, and the second in $W$.) But the properties of the dot product then mean that $x$ is the zero vector.

There is a further concept:

\begin{definition}
Let $V$ be a vectors subspace of $\mathbb{R}^n$. The \emph{orthogonal complement} of $V$ is the set
\[
V^{\perp} = \{ w \in \mathbb{R}^n \mid w \cdot v = 0 \text{ for all } v \in V \}.
\]
\end{definition}

The basic idea is that two spaces are orthogonal complements if they are orthogonal, and together they contain enough vectors to span the entire space. The definition looks like it is a one-directional thing: for a subspace, you find its orthogonal complement. But really it is a \underline{complementary} relationship. If $W$ is the orthogonal complement to $V$, then $V$ is the orthogonal complement to $W$.



Recall the four fundamental subspaces associated to an $m\times n$ matrix $A$.
\begin{itemize}
\item The column space, $\mathrm{col}(A)$, spanned by all of the columns of $A$. This is a subspace of $\mathbb{R}^m$.
\item The row space, $\mathrm{row}(A)$, spanned by all of the rows of $A$. This is a subspace of $\mathbb{R}^n$. This also happens to be the column space of $A^T$.
\item The nullspace (or kernel), $\mathrm{null}(A)$, consisting of all those vectors $x$ for which $Ax=0$. This is a subspace of $\mathbb{R}^n$.
\item The left nullspace, which is just the nullspace of $A^T$. This is a subspace of $\mathbb{R}^m$.
\end{itemize}

And we have another big result:

\begin{theorem}
If $A$ is an $m\times n$ matrix, then
\begin{compactitem}
\item The nullspace of $A$ and the row space of $A$ are orthogonal complements of one another.
\item The column space of $A$ and the left nullspace of $A$ are orthogonal complements of one another.
\end{compactitem}
\end{theorem}


\section{Sage instructions}

I have made a Sage worksheet file with some basic commands that you might find useful in investigating matrices. The file is called \texttt{section4\_1.sagews}.


\section{Questions for Section 4.1}
\setcounter{exercise}{116}

\begin{exercise} Do Strang Exercise 4.1 number 2. \end{exercise}
\begin{exercise} Do Strang Exercise 4.1 number 3. \end{exercise}
\begin{exercise} Do Strang Exercise 4.1 number 4. \end{exercise}
\begin{exercise} Do Strang Exercise 4.1 number 9. \end{exercise}
\begin{exercise} Do Strang Exercise 4.1 number 12. \end{exercise}
\begin{exercise} Do Strang Exercise 4.1 number 22. \end{exercise}


\end{document}




%sagemathcloud={"zoom_width":100}