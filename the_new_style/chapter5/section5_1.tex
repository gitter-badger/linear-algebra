\documentclass[11pt]{amsart}
\usepackage[margin=1in]{geometry}
\usepackage{paralist}
\usepackage{graphicx}
\usepackage{mathtools}

\theoremstyle{definition}
\newtheorem{exercise}{Exercise}
\newtheorem*{theorem}{Theorem}
\newtheorem*{definition}{Definition}

\begin{document}
\title{Linear Algebra}
\author{Strang, Section 5.1}
\maketitle

\section{The assignment}
\begin{compactitem}
\item Read section 5.1 of Strang (pages 244-251).
\item Read the following.
\item Prepare the items below for presentation.
\end{compactitem}


\section{The Determinant}

Generally, the most interesting matrices to look at are the square ones. For square matrices, there is an important number called the \emph{determinant} which helps us determine if the matrix is invertible or not.

Strang lists 10 important properties of determinants in this section, and verifies them for $2\times 2$ matrices. The verifications for general matrices aren't any harder, but they sure are \textbf{longer}, so I am glad he skipped them. Anyway, these properties are enough to get by when it is time to compute. In fact, clever use of these properties can save you a lot of time.

\subsection{Interpretation}

Suppose that an $n\times n$ matrix $A$ is represented as a collection of its column vectors:
\[
A = \begin{pmatrix}
| & | &  & | \\
v_1 & v_2 & \dots & v_n \\
| & | &  & |
\end{pmatrix} .
\]
Then the geometric significance of the determinant is this: The number $\det(A)$ represents the \emph{signed} $n$-dimensional volume of the $n$-dimensional box in $\mathbb{R}^n$ with sides $v_1, v_2, \ldots, v_n$.

This takes a bit of getting used to, and the hardest part is the choice of signs. We choose a positive sign if the vectors $v_1, v_2, \ldots, v_n$ have the same orientation as the standard basis.

\subsection{The Importance of the Determinant}

The real importance of the determinant is described in the following theorem. Note that this is a special result for \emph{square} matrices. The shape is crucial for this result.

\begin{theorem}[The Invertible Matrix Theorem]
Let $A$ be an $n\times n$ matrix. Then the following conditions are equivalent:
\begin{itemize}
\item The columns of $A$ are linearly independent.
\item The columns of $A$ are a spanning set for $\mathbb{R}^n$.
\item The colums of $A$ are a basis for $\mathbb{R}^n$.
\item The rows of $A$ are linearly independent.
\item The rows of $A$ are a spanning set for $\mathbb{R}^n$.
\item The rows of $A$ are a basis for $\mathbb{R}^n$.
\item For any choice of vector $b \in \mathbb{R}^n$, the system of linear equations $Ax = b$ has a unique solution.
\item $A$ is invertible.
\item The transpose $A^T$ is invertible.
\item $\det(A) \neq 0$.
\item $\det(A^T) \neq 0$.
\end{itemize}
\end{theorem}



\section{Sage instructions}

I have made a Sage worksheet file with some basic commands that you might find useful. The file is called \texttt{section5\_1.sagews}.


\section{Questions for Section 5.1}
\setcounter{exercise}{135}



\begin{exercise}
Exercises 1 \& 2 from section 5.1 of Strang.
\end{exercise}

\begin{exercise}
Exercise 12 from section 5.1 of Strang.
\end{exercise}

\begin{exercise}
Exercise 14 from section 5.1 of Strang.
\end{exercise}

\begin{exercise}
Exercise 15 from section 5.1 of Strang.
\end{exercise}

\begin{exercise}
Exercise 23 from section 5.1 of Strang.
\end{exercise}

\end{document}




%sagemathcloud={"zoom_width":100}