\documentclass[11pt]{amsart}
\usepackage[margin=1in]{geometry}
\usepackage{paralist}
\usepackage{graphicx}
\usepackage{mathtools}

\theoremstyle{definition}
\newtheorem{exercise}{Exercise}
\newtheorem*{theorem}{Theorem}
\newtheorem*{definition}{Definition}

\begin{document}
\title{Linear Algebra}
\author{Strang, Section 6.1}
\maketitle

\section{The assignment}
\begin{compactitem}
\item Read section 6.1 of Strang (pages 283-292).
\item Read the following.
\item Prepare the items below for presentation.
\end{compactitem}


\section{Eigenvalues and Eigenvectors}

We have discussed the ``transformational view'' of the geometry of a system of $m$ linear equations $Ax = b$ in $n$ unknowns $x$, where we view the $m \times n$ matrix $A$ as defining
a function from $\mathbb{R}^m$ to $\mathbb{R}^n$. In the case of a square matrix $m=n$, the domain and the target are the same space $\mathbb{R}^n$. So we can think of $A$ as making a function from one space to itself.

This means it might be interesting to think about how $A$ moves a vector about inside of $\mathbb{R}^n$. Usually, the vector $v$ will get turned into a vector $Av$ which has a different length and points in a completely different direction. But sometimes, \emph{sometimes}, $v$ and $Av$ will point in the same direction. This is an eigenvector.

A number $\lambda$ is called an eigenvalue of the matrix $A$ when the matrix $A-\lambda I$ is singular. A vector $v$ is called an eigenvector of $A$ corresponding to $\lambda$ when $v$ is not zero but still lies in the null space of $A-\lambda I$. We exclude $0$ from being an eigenvector because it is boring. The zero vector lies in every subspace, including the nullspace of any matrix.

As Strang discusses, the eigenvalues are found as roots of the \emph{characteristic polynomial} $\det(A-\lambda \cdot I) = 0$. That's right, we only need to find the roots of a polynomial! Sounds great, but as a general thing this is pretty hard. Don't get too excited. Have you heard this fact before: depressing and interesting: there is no general formula to find the roots of a polynomial of degree 5 or more. 

\section{Sage instructions}

I have made a Sage worksheet file with some basic commands that you might find useful. The file is called \texttt{section6\_1.sagews}.


\section{Questions for Section 6.1}
\setcounter{exercise}{140}



\begin{exercise}
Exercise 5 from section 6.1 of Strang.
\end{exercise}

\begin{exercise}
Exercise 6 from section 6.1 of Strang.
\end{exercise}

\begin{exercise}
Exercise 7 from section 6.1 of Strang.
\end{exercise}

\begin{exercise}
Exercise 12 from section 6.1 of Strang.
\end{exercise}

\begin{exercise}
Exercise 14 from section 6.1 of Strang.
\end{exercise}

\begin{exercise}
Exercise 19 from section 6.1 of Strang.
\end{exercise}



\end{document}




%sagemathcloud={"zoom_width":100}