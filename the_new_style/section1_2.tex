\documentclass[11pt]{amsart}
\usepackage{paralist}
\usepackage[margin=1in]{geometry}

\theoremstyle{definition}
\newtheorem{exercise}{Exercise}


\begin{document}

\title{Linear Algebra}
\author{Section 1.2}

\maketitle

\section{The Assignment}
\begin{compactitem}
\item Read \emph{Strang} Chapter 1, section 2, pages 11-18.
\item (Optional) Read \emph{Hefferon} Chapter One, Part II Linear Geometry, section 2 Length and Angle Measures, pages 39-43.
\item Complete the exercises below before class.
\end{compactitem}

\section{Discussion}
The dot product is a wonderful tool for encoding the geometry of Euclidean space, but it can be a bit mysterious at first. As \emph{Strang} shows, it somehow holds all of the information you need to measure lengths and angles.

The connection to linear algebra comes about through this: a dot product with a ``variable vector'' is a way of writing a linear equation. For example,
\begin{equation*}
\begin{pmatrix} 7 \\ 3 \\ -2 \end{pmatrix} \cdot
\begin{pmatrix} x \\ y \\ z \end{pmatrix} = 7x + 3y -2z.
\end{equation*}
Sometimes this will allow us to connect linear algebra to geometry, and use geometric thinking to answer algebraic questions.

\section{Sage Instructions}

I have made a Sage .sagews file called \emph{section1-2.sagews} which contains some basic commands for dealing with vectors and dot products, and a couple of interactive demos to play with dot products in two and three dimensions.

\section{The Exercises}

\setcounter{exercise}{12}

\begin{exercise}
What shape is the set of solutions $\left(\begin{smallmatrix} x \\ y\end{smallmatrix}\right)$ to the equation
\begin{equation*}
\begin{pmatrix} 3 \\ 7 \end{pmatrix} \cdot \begin{pmatrix} x \\ y \end{pmatrix} = 5?
\end{equation*}
That is, if we look at all possible vectors $\left( \begin{smallmatrix} x \\ y \end{smallmatrix} \right)$ which make the equation true, what shape does this make in the plane?

What happens if we change the vector $\left(\begin{smallmatrix} 3 \\ 7\end{smallmatrix}\right)$ to some other vector? What happens if we change the number $5$ to some other number?

What happens if we move up a dimension, and instead work in three dimensional space?
\end{exercise}

\begin{exercise}
Find an example of two $2$-vectors $v$ and $w$ such that $\left(\begin{smallmatrix} 1 \\ 2 \end{smallmatrix}\right) \cdot v =0$ and $\left(\begin{smallmatrix} 1 \\ 2 \end{smallmatrix}\right) \cdot w = 0$ and $v\cdot w = 0$, or explain why such an example is not possible.
\end{exercise}

\begin{exercise}
Let $v = \left( \begin{smallmatrix} 3 \\ -1 \end{smallmatrix} \right)$. Find an example of a pair of vectors $u$ and $w$ such that $v \cdot u < 0$ and $v \cdot w < 0$ and $w\cdot u =0$, or explain why no such pair of vectors can exist.
\end{exercise}

\begin{exercise}
Find an example of three $2$-vectors $u$, $v$, and $w$ so that $u\cdot v < 0$ and $u \cdot w < 0$ and $v \cdot w < 0$, or explain why no such example exists.
\end{exercise}

\begin{exercise}
Find an example of a number $c$ so that
\begin{equation*}
\begin{pmatrix} 1 \\ -1 \end{pmatrix} \cdot \begin{pmatrix} x \\ y \end{pmatrix} = c
\end{equation*}
has the vector $\left( \begin{smallmatrix} 4 \\ 7 \end{smallmatrix} \right)$ as a solution, or explain why no such number exists.
\end{exercise}

\begin{exercise}
Let $v = \left(\begin{smallmatrix} 2 \\ 1 \end{smallmatrix}\right)$ and $w = \left(\begin{smallmatrix} -3 \\ 4 \end{smallmatrix} \right)$. Find an example of a number $c$ so that
\begin{equation*}
v \cdot \begin{pmatrix} 1 \\ -1 \end{pmatrix} = c, \text{ and } w \cdot \begin{pmatrix} 1 \\ -1 \end{pmatrix} = c,
\end{equation*}
or explain why this is not possible.

\end{exercise}

\begin{exercise}
Let $P = \begin{pmatrix}-3 \\ 4 \end{pmatrix}$. Find an example of numbers $c$ and $d$ so that
\begin{equation*}
\begin{pmatrix} 2 \\ -1 \end{pmatrix} \cdot P = c, \text{ and }
\begin{pmatrix} 1 \\ -1 \end{pmatrix} \cdot P = d,
\end{equation*}
or explain why no such example is possible.
\end{exercise}

\textbf{Now, we move to three dimensions\dots}

\begin{exercise}
Let $V = \left(\begin{smallmatrix} 1 \\ 1 \\ 1 \end{smallmatrix}\right)$. Find an example of two vectors $U$ and $W$ such that
\begin{equation*}
U\cdot V < 0, \qquad U \cdot W < 0, \qquad \text{and } V \cdot W < 0,
\end{equation*}
or explain why no such example exists.
\end{exercise}

\begin{exercise}
Let $V = \left(\begin{smallmatrix} 1 \\ 1 \\ 1 \end{smallmatrix}\right)$. Find a unit vector of the form $X = \left(\begin{smallmatrix} x \\ y \\ 0 \end{smallmatrix}\right)$ so that $V \cdot X = \sqrt{2}$, or explain why no such vector exists.
\end{exercise}

\begin{exercise}
Let $V = \left(\begin{smallmatrix} 1 \\ 1 \\ 1 \end{smallmatrix}\right)$. Find a unit vector of the form $X = \left(\begin{smallmatrix} x \\ y \\ 0 \end{smallmatrix}\right)$ so that $V \cdot X = 10$, or explain why no such vector exists.
\end{exercise}

\begin{exercise}
Find an example of numbers $c$, $d$, and $e$ so that there is no solution vector $X = \left( \begin{smallmatrix} x \\ y \\ z \end{smallmatrix} \right)$ which simultaneously satisfies the three equations
\begin{equation*}
\begin{pmatrix} 1 \\ 1 \\ 1 \end{pmatrix} \cdot X = c, \qquad
\begin{pmatrix} 2 \\ 2 \\ 2 \end{pmatrix} \cdot X = d, \qquad
\begin{pmatrix} 0 \\ 0 \\ 1 \end{pmatrix} \cdot X = e,
\end{equation*}
or explain why no such numbers exist.
\end{exercise}

\begin{exercise}
Find an example of numbers $c$, $d$, and $e$ so that there is no solution vector $X = \left( \begin{smallmatrix} x \\ y \\ z \end{smallmatrix} \right)$ which simultaneously satisfies the three equations
\begin{equation*}
\begin{pmatrix} 1 \\ 1 \\ 1 \end{pmatrix} \cdot X = c, \qquad
\begin{pmatrix} 0 \\ 1 \\ 1 \end{pmatrix} \cdot X = d, \qquad
\begin{pmatrix} 0 \\ 0 \\ 1 \end{pmatrix} \cdot X = e,
\end{equation*}
or explain why no such numbers exist.
\end{exercise}


\end{document}

%sagemathcloud={"zoom_width":100,"latex_command":"pdflatex -synctex=1 -interact=nonstopmode section1-2.tex"}








